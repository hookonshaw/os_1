\section{Условие}
        Родительский процесс создает дочерний процесс. Первой строчкой пользователь в консоль
родительского процесса пишет имя файла, которое будет передано при создании дочернего
процесса. Родительский и дочерний процесс должны быть представлены разными программами.
Родительский процесс передает команды пользователя через pipe1, который связан с
стандартным входным потоком дочернего процесса. Дочерний процесс при необходимости
передает данные в родительский процесс через pipe2. Результаты своей работы дочерний
процесс пишет в созданный им файл. Допускается просто открыть файл и писать туда, не
перенаправляя стандартный поток вывода.

\subsection*{Цель работы}
        Изучение принципов создания дочерних процессов и организации межпроцессного взаимодействия с использованием механизма pipes в операционной системе.
\subsection*{Задание}
        Пользователь вводит команды вида: «число число число<endline>». Далее эти числа передаются от родительского процесса в дочерний. Дочерний процесс производит деление первого числа, на последующие, а результат выводит в файл.
Если происходит деление на 0, то тогда дочерний и родительский процесс завершают свою работу. Проверка деления на 0 должна осуществляться на стороне дочернего процесса. 
Числа имеют тип float. Количество чисел может быть произвольным.

\subsection*{Вариант}4