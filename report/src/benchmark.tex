\section{Результаты}

В результате работы была разработана многопроцессная система для выполнения последовательных операций деления с обработкой ошибок. Программа успешно решает поставленную задачу и обладает следующими ключевыми особенностями:

\subsection*{Основные достижения}
\begin{itemize}
    \item Реализовано корректное межпроцессное взаимодействие через два pipe-канала
    \item Обеспечена двусторонняя связь между родительским и дочерним процессами
    \item Реализована надежная обработка ошибки деления на ноль с использованием механизма сигналов
    \item Организован корректный обмен данными произвольного количества чисел типа float
\end{itemize}

\subsection*{Особенности реализации}
\begin{itemize}
    \item \textbf{Модульная архитектура} - четкое разделение на родительский и дочерний процессы, представленные разными исполняемыми файлами
    \item \textbf{Надежная обработка ошибок} - при обнаружении деления на ноль оба процесса корректно завершают работу
    \item \textbf{Гибкий ввод данных} - поддержка произвольного количества чисел в одной операции
\end{itemize}

\subsection*{Протокол взаимодействия}
Программа демонстрирует эффективный протокол обмена данными:
\begin{enumerate}
    \item Передача имени файла для результатов
    \item Последовательная отправка наборов чисел с указанием их количества
    \item Получение подтверждения о успешном выполнении операции или ошибке
    \item Корректное завершение при получении команды или критической ошибки
\end{enumerate}

\subsection*{Обработка исключительных ситуаций}
\begin{itemize}
    \item Контроль деления на ноль на стороне дочернего процесса
    \item Проверка минимального количества чисел (не менее 2)
    \item Обработка ошибок открытия файла и работы с pipe-каналами
    \item Корректное освобождение ресурсов при завершении работы
\end{itemize}